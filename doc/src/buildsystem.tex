
\newpage
\section{Seawolf Build System} \label{build}
In lieu of a Make based build system, Seawolf's software stack is built with a
simple collection of shell scripts which emulate a number of Make-like features.

\subsection{Using the Build System} \label{buildusing}
In directories such as \folder{applications}, \folder{doc}, \folder{serial}, and
\folder{libseawolf} you will find shell scripts called \file{make}, and each of
these is the build script for their respective component. To start a build,
begin by nagivating to the given directory and then executing
\texttt{./make}. This should generate all output files for the given
component. If you instead execute \texttt{./make clean} the script should remove
all output files.

\subsection{Writing a build script} \label{buildwriting}
To write a build script begin with the following template which is canonically
named \file{make}.

\begin{lstlisting}[language=sh, caption=Build script template]
#!/bin/sh

CC=``/usr/bin/gcc''
CFLAGS=``--std=c99 -Wall -pedantic -g''

source ../misc/make.sh

case ``$1'' in
  ``clean'')
    # Remove files
    ;;
  *)
    # Build files 
    ;;
esac
\end{lstlisting}

More targets be can added as desired by introducing new branches to the main
case statement. Functions included from \file{misc/make.sh} can make writing
these targets easier. 

The function \func{runcmd} takes a program to run as its argument and will print
the program and its argumets before actually running it. This should be used to
provide visual feedback for all commands which are executed.

The function \func{isOutDated} can be used to determine if an output file needs
to be updated. The first argument is the output file name and the rest of the
arguments are the source files that the output file is made from. If any of the
dependencies have more recent modification times then the output file then the
output file is considered out-of-date and the function returns true. If the
output file is the most recently modified then it is up-to-date and the function
returns false.

The function \func{rmNoFail} will remove a file if it exists and return
immediadately otherwise. This can be used to write simple ``clean'' targets. 

The following is an example make script. This one in paricular builds the
applications.

\begin{lstlisting}[language=sh, caption=Applications build script]
#!/bin/sh

CC="/usr/bin/gcc"
CFLAGS="--std=c99 -Wall -pedantic"
INCLUDES="-I../libseawolf/include"
LIBFLAGS="-g -lncurses -lrt -lseawolf -lpthread -L../libseawolf/"
OUTPUTDIR="bin"

source ../misc/make.sh

case "$1" in
  "clean")
    rmNoFail ${OUTPUTDIR}/*
    ;;
  *)
    for in in src/*.c; do
        out="${OUTPUTDIR}/`basename ${in} .c`"
        if isOutDated $out $in; then
            runcmd ${CC} ${CFLAGS} ${INCLUDES} -o $out $in ${LIBFLAGS}
        fi
    done;
esac
\end{lstlisting}
