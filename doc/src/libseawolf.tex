
\newpage
\section{Seawolf Library - \libseawolf{}} \label{libseawolf}
\subsection{Component Overview} \label{libseawolfcomponents}
\subsubsection{SeaSQL} \label{overviewseasql}
SeaSQL (pronounced see-squel) is the database communications component of
\libseawolf. It transparently manages all components of database interfacing and
provides a simple and easy to use set of functions to the application programmer
for setting and retrieving variable values from the MySQL database. SeaSQL also
provides an interface for writing extension libraries such as the Logging
components of \libseawolf{}. SeaSQL supports automatically interacting with the
Notify (see \interlink{overviewnotify}{Notify}) component of \libseawolf{} in
order to notify applications of variables updates.

\subsubsection{Logging} \label{overviewlogging}
\libseawolf{} also provides a standard logging interface that interacts with
SeaSQL to write logging and debugging information to the MySQL database. This
logging information can then accessed by logging into the MySQL server and
browsing the 'log' table in the database. This can be done with the command line
utility \texttt{mysql} or a web-based utility such as phpMyAdmin. Graphical
tools for Microsoft Windows operating systems are available,
\url{http://dev.mysql.com/downloads/gui-tools/5.0.html}. The Logging components
can also be configured to output message to the standard output to assist in
real time debugging.

\subsubsection{Serial} \label{overviewserial}
The serial component provides a set of convenience functions for interacting
with serial devices under Linux. It provides functions to open and close serial
ports, set serial port settings, and high level functions for reading from and
writing to serial devices. Functions in this component should only ever be
directly accessed by serial drivers.

\subsubsection{Notify} \label{overviewnotify}
Notify is a component of \libseawolf{} critical to maintaining quick and
responsive applications and avoiding having applications spending excessive
amounts of time simply polling the database for new data. Notify allows one
application to notify others of new data of interest and also allows other
applications to receive these notifications and know when new data is
available. Notify runs in a server/client model, where the server acts as a hub
or repeater between the clients (see \interlink{overviewhub}{The Hub
  Server}). When one client sends a message, all other clients are then sent
that message by the hub server. In practice this means that when one application
makes new data available through SeaSQL all other applications will be notified
of its availability. SeaSQL utilizes Notify in order to automatically make the
availability of new data known to other clients, thus relieving the application
programmer of this task. Notify messages have a standard format,
\vspace{10pt}\hfill\newline
\indent\texttt{ACTION} \textit{param1 param2 \ldots{}} \texttt{$\backslash$0}
\vspace{10pt}\hfill\newline \texttt{ACTION} is a word words such as
\texttt{UPDATED}, or \texttt{EXECUTED} which describes what event
occurred. Parameters follow the action and are ASCII strings with spaces between
each parameter. Actions and parameters are case sensitive, but actions should
semantically be all uppercase. Message are then terminated with the null value
\texttt{$\backslash$0}.

\subsubsection{Utilities} \label{overviewutil}
The utilities (Util) section of \libseawolf{} provides a number of useful, high
level functions which may be useful to application developers. If an application
developer finds him or herself needing a generally useful function they should
check to see if it is included in the Util section, and if not it might be good
to include it.

\subsubsection{Timer} \label{overviewtimer}
This component provides instances of objects called ``Timers''. A Timer is able
to keep track of running time as well as provide time deltas for reoccuring
events. These Timers are well suited to writing PIDs and were written for such a
purpose.

\subsubsection{PID} \label{overviewpid}
The PID component provides an easy to use implementation of a generic PID
controller.

\subsubsection{Stack} \label{overviewstack}
An implementation of a stack-like data structure is provided for general use.

\subsubsection{Task} \label{overviewtask}
Provides basic job scheduling and queueing through a simple interface.

\subsubsection{ArdComm} \label{overviewardcomm}
ArdComm is a useful set of functions to provide high level communication with
serial devices which support the ArdComm protocol. This component was designed
with Arduinno development board in mind, but can easily be used with any serial
device supporting the protocol. This protocal is described in the ardcomm.h
header file. See this file for any additional required information. \textbf{This
  component is largely depricated and only the handshake code is still in use.}

\subsubsection{The Hub Server} \label{overviewhub}
The hub server is not in fact a component of \libseawolf{} itself but is actual a
standalone program that is bundled with \libseawolf{}. The hub server works with
the Notify (see \interlink{overviewnotify}{Notify}) component to relay messages
between clients. When properly used, any message sent through Notify will be
received by a currently running hub server. The hub server will then take this
message and send it back out to all other connected clients. In this way it acts
very similarly to a hub or repeater.

\subsection {Using \libseawolf} \label{usinglibseawolf}
\libseawolf{} is provided as a shared library along with a set of header files
 which provide definitions for functions available
through \libseawolf{}. Writing applications is as simple as placing your
application in a predefined location and it will automatically be integrated
by the build system.

\subsubsection{The Environment} \label{environment}
All general purpose Seawolf applications are located under a single directory in
the overall Seawolf directory structure outlined below. By placing the source
file for your application in \folder{seawolf3/applications/src} it can easily be
built with the rest of the general purpose Seawolf applications.

\begin{lstlisting}[language=FileList, caption=Directory structure, escapeinside={{*@}{@*}}]
seawolf3/          /* The root directory. CODE2009 in the SVN */
  applications/
    +Makefile       /* Applications build script */
    bin            /* Binaries placed here by the build system */
    src/           /* Source files for applications */
  
  serial/          /* Serial code */
    *@\ldots{}@*
 
  libseawolf/      /* libseawolf code */
    +Makefile
    src/
    include/
      +seawolf.h
      seawolf/
      *@\ldots{}@*
\end{lstlisting}

\subsubsection{A Basic Seawolf Application} \label{environmentbasicapp}
\libseawolf{} is meant to easily integrated with your application and not get in
the way as much as is possible. The library does impose a few simple
requirements that are outlined in the folloiwng example program.

\lstinputlisting[language=Cextended, caption=Seawolf Application Outline]{src/code_samples/app_outline.c}

\par The include \texttt{seawolf.h} automatically pulls in all \libseawolf{}
header files and makes all functions definitions available to the
program. Components may be included individually, but simply including
\texttt{seawolf.h} is far simpler and doesn't significantly affect program size
or execution speed.

The \inlinelink{apiseawolfinit}{\func{Seawolf\_init}} and
\inlinelink{apiseawolfclose}{\func{Seawolf\_close}} function calls are necessary
for proper program execution. The only code that should go before the
\inlinelink{apiseawolfinit}{\func{Seawolf\_init}} are calls to functions to set
options or settings within \libseawolf{} necessary before the library is
initiated. When \inlinelink{apiseawolfinit}{\func{Seawolf\_init}} is called,
\libseawolf{} will open a connection to the MySQL database and initialize all
other subcomponents. \inlinelink{apiseawolfclose}{\func{Seawolf\_close}} does
the opposite, tying up loose ends, and closing database connections and serial
devices before the program exits. Since
\inlinelink{apiseawolfclose}{\func{Seawolf\_close}} closes all database
connections and shuts down all \libseawolf{} components, no further
\libseawolf{} calls should be made after this point. In the event of an
application being killed before
\inlinelink{apiseawolfclose}{\func{Seawolf\_close}} can be called by normal
means, \libseawolf{} will execute it automatically to ensure clean shutdown.

With these basic dependencies satisfied an application can be developed as
normal and actively utilize any \libseawolf{} functions. These functions are
documented in \interlink{api}{API and Function Reference}.
