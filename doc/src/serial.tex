
\newpage
\section{Serial Management} \label{serial}
As described in \inlinelink{serialarch}{Serial Management} in the component
overview, the serial managment applications are a set of programs which serve to
act as an abstraction layer between the physical IO devices and normal
applications. The serial management code handles identifying serial devices,
fingerprinting them to determine their device type, and spawning appropriate
driver applications to manage the device after the point of
identification. Below is the organization of the files that form the serial
management code.

\begin{lstlisting}[language=FileList, caption=Serial directory structure, escapeinside={{*@}{@*}}]
serial/
  +Makefile
  bin/
    +serialapp
    +altimeter
    +depth-bin
    *@\ldots{}@*
  src/
    +Makefile
    +serialapp.c
    drivers/
      ard/
        +ard.c
        +depth-bin.c
        +thruster12-bin.c
        *@\ldots{}@*
      +altimeter.c
      +imu.c
      *@\ldots{}@*
\end{lstlisting}

When \file{serialapp} opens a device it attempts to ``fingerprint'' it in order
to determine its device type. Once a device is identified the serial port is
closed and the correct driver from is spawned with the device path of the
correct serial port passes as the only argument. It is then up to the individual
driver to handle the device, and \file{serialapp} will no longer attempt to
handle it.
