\newpage
\section{API and Function Reference} \label{api}
This section covers all functions targeted at application developers as well as
those functions used by the serial components. Functions only used internally by
\libseawolf{} are documented in the sections on extending and modifying
\libseawolf{} (see \interlink{extending}{Extending the Seawolf Software Stack}).

\subsection{Top Level} \label{apitop}
\subsubsection{Seawolf\_init} \label{apiseawolfinit}
\cdoc{void Seawolf_init(const char* name)} Initialize all \libseawolf{} sub
components. Only configuration related function calls should be made before this
is executed. \texttt{name} should be a descriptive name for the application.

\subsubsection{Seawolf\_close} \label{apiseawolfclose}
\cdoc{void Seawolf_close(void)} Close all \libseawolf{} components and close
network connections and free memory. This should always be the last
\libseawolf{} function call made.

\subsubsection{Seawolf\_getName} \label{apiseawolfgetname}
\cdoc{char* Seawolf_getName(void)} Return the name of this application as passed
to \inlinelink{apiseawolfinit}{\func{Seawolf\_init}}.

\subsubsection{Seawolf\_loadConfig} \label{apiseawolfconfig}
\cdoc{void Seawolf_loadConfig(const char* filename)} Load a configuration file
for \libseawolf{} with options to configure all subcomponents. The format of
this configuration file is documented in the
\inlinelink{formatsseawolfconf}{Seawolf Configuration File} section.

\subsection{SeaSQL} \label{apiseasql}
\subsubsection{SeaSQL\_get\bracket{var}} \label{apiseasqlget}
\cdoc{float SeaSQL_get<var>(void)} Retrieve the most recent value of the
variable \bracket{var} as a floating point value.  Possible variables are listed
below. If additional variables are needed, they should be added directly to
\libseawolf{} (see \interlink{extendingseasql}{Extending SeaSQL}) and documented here.
\begin{center}
\begin{tabular}{l l}
MissionID & MissionStatus
\end{tabular}\vspace{5pt}

\begin{tabular}{l l l l l}
Aft & PortX & PortY & StarX & StarY
\end{tabular}\vspace{5pt}

\begin{tabular}{l l l l}
Depth & DepthHeading & Altitude & AltitudeHeading \\
\end{tabular}\vspace{5pt}

\begin{tabular}{l l l}
SEA\_Roll & SEA\_Yaw & SEA\_Pitch \\
IEA\_Roll & IEA\_Yaw & IEA\_Pitch \\
IV\_MagField\_X & IV\_MagField\_Y & IV\_MagField\_Z \\
IV\_Accel\_X & IV\_Accel\_Y & IV\_Accel\_Z \\
IV\_AngRate\_X & IV\_AngRate\_Y & IV\_AngRate\_Z \\
IOM\_11 & IOM\_21 & IOM\_31 \\
IOM\_12 & IOM\_22 & IOM\_32 \\
IOM\_13 & IOM\_23 & IOM\_33 \\
\end{tabular}
\end{center}

\subsubsection{SeaSQL\_set\bracket{var}} \label{apiseasqlset}
\cdoc{int SeaSQL_set<var>(float value)} Set the value of the variable whose name
is \bracket{var} to the floating point value given by \texttt{value}. This
function returns 0 to indicate success, any other value indicates error. See
\interlink{apiseasqlget}{SeaSQL\_get\bracket{var}} for the list of valid
values for \bracket{var}.

\subsubsection{SeaSQL\_setAutoNotify} \label{apiseasqlnotify}
\cdoc{void SeaSQL_setAutoNotify(bool do_notify)} Set whether SeaSQL should
automatically send out a Notify \texttt{UPDATED} message (see
\interlink{overviewnotify}{Notify}) with every call to
\inlinelink{apiseasqlset}{\func{SeaSQL\_set\bracket{var}}}. By default this is true,
but certain applications may prefer to send out these message manually to
control when they are sent.

\subsection{Logging} \label{apilogging}
\subsubsection{Logging\_log} \label{apilogginglog}
\cdoc{void Logging_log(short level, char* message)} Log the string stored in
\texttt{message} to the log table of the SeaSQL database and associates this log
item with the severity level given by \texttt{level}. Possible levels include
\texttt{DEBUG}, \texttt{INFO}, \texttt{NORMAL}, \texttt{WARNING},
\texttt{ERROR}, and \texttt{CRITICAL}. This message will not be logged if the
level given is lower than the minimum logging level. The minimum logging level
is \texttt{NORMAL} by default and can be changed through
\inlinelink{apiloggingsetthreshold}{\func{Logging\_setThreshold}}.

\subsubsection{Logging\_setThreshold} \label{apiloggingsetthreshold}
\cdoc{void Logging_setThreshold(short level)} Set the minimum level of log
message to write to the database. See \interlink{apilogginglog}{Logging\_log}
for possible values of \texttt{level}. This can also be set in the
\inlinelink{formatsseawolfconf}{Seawolf Configuration File}. 

\subsubsection{Logging\_replicateStdio} \label{apiloggingreplicate}
\cdoc{void Logging_replicateStdio(bool do_replicate)} Set whether logged message
should be written to the standard output as well as to the database. This can
also be set in the \inlinelink{formatsseawolfconf}{Seawolf Configuration File}.


\subsection{Serial} \label{apiserial}
\subsubsection{Serial\_open} \label{apiserialopen}
\cdoc{SerialPort Serial_open(const char* device_path)} Open the specified device
and return a handle to it that can be passed to future Serial calls. As well as
opening the device, a number of standard options are set on the port to ready it
for communication. In the event that an error occurs while attempting to open
the device the value 0 will be returned. 

\subsubsection{Serial\_closePort} \label{apiserialclose}
\cdoc{int Serial_closePort(SerialPort sp)} Close the specified serial
device. After this call the serial device can not be passed to any future Serial
calls. This is executed for all devices left open at the time of
\inlinelink{apiseawolfclose}{\func{Seawolf\_close}} being called to ensure that
devices are not left open when the application terminates.

\subsubsection{Serial\_setBlocking} \label{apiserialblocking}
\cdoc{void Serial_setBlocking(SerialPort sp)} Set the specified serial port to
be placed into blocking mode. When in blocking mode, calls to any
\func{Serial\_get*} functions will not return until data is available. If a port
is in non-blocking mode then these functions will return immediately if data is
not already available.

\subsubsection{Serial\_setNonBlocking} \label{apiserialnonblocking}
\cdoc{void Serial_setNonBlocking(SerialPort sp)} Set the specified serial port
to be placed into non-blocking mode. See
\interlink{apiserialblocking}{\func{Serial\_setBlocking}}.

\subsubsection{Serial\_getByte} \label{apiserialgetbyte}
\cdoc{int Serial_getByte(SerialPort sp)} Return the next byte in the incoming
buffer for the specified serial port. If the port is in non-blocking mode (see
\interlink{apiserialblocking}{\func{Serial\_setBlocking}}) and no data is
available -1 is returned. 

\subsubsection{Serial\_get} \label{apiserialget}
\cdoc{void Serial_get(SerialPort sp, void* buffer, size_t count)} Read
\texttt{count} bytes from the serial port into the buffer pointed to by
\texttt{buffer}. This function is guaranteed to return the requested number of
bytes and is therefore incompatible with non-blocking mode (see
\interlink{apiserialblocking}{\func{Serial\_setBlocking}}).

\subsubsection{Serial\_getLine} \label{apiserialgetlin}
\cdoc{void Serial_getLine(SerialPort sp, char* buffer)} Read characters from the
serial buffer into the buffer pointed to by \texttt{buffer} until a newline is
reached. This assumes that the serial device is using a human readable protocol
and is in non-blocking mode (see
\interlink{apiserialblocking}{\func{Serial\_setBlocking}}).

\subsubsection{Serial\_sendByte} \label{apiserialsendbyte}
\cdoc{void Serial_sendByte(SerialPort sp, unsigned char b)} Send the single byte
\texttt{b} using the given serial port.

\subsubsection{Serial\_send} \label{apiserialsend}
\cdoc{void Serial_send(SerialPort sp, void* buffer, size_t count)} Send
\texttt{count} bytes from the buffer pointed to by \texttt{buffer} using the
given serial port.


\subsection{ArdComm} \label{apiardcomm}
\subsubsection{ArdComm\_getId} \label{apiardcommgetid}
\cdoc{int ArdComm_getId(SerialPort sp, char* id)} Attempt to read the identifier
for a serial device compliant with the ArdComm protocol (see
\interlink{formatsardcomm}{ArdComm Serial Protocol}) attached to the given
serial port. If an error is encountered while attempting to obtain the
identifier then -1 will be returned. Upon success, 0 will be returned and the
identifier stored in \texttt{id}.

\subsubsection{ArdComm\_handshake} \label{apiardcommhandshake}
\cdoc{void ArdComm_handshake(SerialPort sp)} Attempt to complete a handshake
with an ArdComm compliant device attached to the given serial port. After this
call both the remote device and the local system are ready for communication.


\subsection{Notify} \label{apinotify}
\subsubsection{Notify\_setMode} \label{apinotifysetmode}
\cdoc{void Notify_setMode(int mode)} Set the IO mode for Notify to
utilize. Possible values are \texttt{NOTIFY\_NET} and
\texttt{NOTIFY\_STDIO}. The default is \texttt{NOTIFY\_NET} which works by
connecting to a ``hub'' server (see \interlink{overviewhub}{The Hub Server}) and
having all message relayed through it. \texttt{NOTIFY\_STDIO} uses standard
input and output to receive and send messages; this is useful for debugging
purposes or running without a hub server. This function can only be called
before initialization functions are called. This can also be set in the
\inlinelink{formatsseawolfconf}{Seawolf Configuration File}.

\subsubsection{Notify\_setServer} \label{apinotifysetserver}
\cdoc{void Notify_setServer(char* hostname)} When using the \texttt{NOTIFY\_NET}
IO mode, Notify needs to connect to a ``hub'' server (see
\interlink{overviewhub}{The Hub Server}) at port 31427 on some network host. The
default host is localhost (127.0.0.1) but this can be overridden by calling this
function and passing a server name as a string in \texttt{hostname}. This
function only has an effect when in \texttt{NOTIFY\_NET} mode and must be called
before initialization functions are called. This can also be set in the
\inlinelink{formatsseawolfconf}{Seawolf Configuration File}.

\subsubsection{Notify\_get} \label{apinotifyget}
\cdoc{void Notify_get(char* action, char* param)} Retrieves the most recent
inbound Notify message or waits for one to become available before
returning. When a message is available which matches a filter (see
\interlink{apinotifyfilter}{Notify\_filter}), it is split and the action is
stored in \texttt{action} and the parameters are stored as a single string in
\texttt{param}.

\subsubsection{Notify\_send} \label{apinotifysend}
\cdoc{void Notify_send(char* action, char* param)} Generates and sends
out a message using the given action and parameter string. The
parameter string should be a single string with all parameters
embedded already.

\subsubsection{Notify\_filter} \label{apinotifyfilter}
\cdoc{void Notify_filter(int filter_type, char* filter)} Enable a new
filter. When a call to \inlinelink{apinotifyget}{\func{Notify\_get}} is made,
the incoming message is checked against all filters. If at least one filter
matches then the message passes through, otherwise it is discarded. If no
filters are present then all messages pass. To remove all filters, call this
function as \lstinline[language=Cextended]!Notify_filter(0, NULL)!.  To add a
new filter, specify a filter type in \texttt{filter\_type} and specify the
filter text in \texttt{filter}. There are three filter types, each of which
perform a different method of filtering. The first filter type is
\texttt{FILTER\_MATCH} which will perform a full message (action and parameters)
match using the filter. If a \texttt{FILTER\_MATCH} filter with text
\lstinline[language=Cextended]!"UPDATED Depth"! is added then only the message
``UPDATED Depth'' will match this filter. The second filter type is
\texttt{FILTER\_ACTION} which performs a match on the action of a
message. Parameters are ignored and the message's action must match perfectly
with the content of the filter. The last type of filter type is
\texttt{FILTER\_PREFIX} which performs a test on the message and matches when
the filter is a substring of the full message which starts with the first
character of the message.

\subsection{Utilities} \label{apiutilities}
\subsubsection{Util\_usleep} \label{apiutilsleep}
\cdoc{void Util_usleep(double s)} Halt execution of the program and sleep for a
number of seconds equal to \texttt{s}. 

\subsubsection{Util\_format} \label{apiutilformat}
\cdoc{char* Util_format(char* format, ...)} Return a formatted string. Consult
documentation for the sprintf function for the syntax for \texttt{format}. This
function is provided as a convience function so formatting may be done in place.

\subsubsection{Util\_strip} \label{apiutilstrip}
\cdoc{void Util_strip(char* buffer)} Remove white space from the beginning end
of the given string. The string is modified in place.

\subsubsection{Util\_split} \label{apiutilsplit}
\cdoc{int Util_split(const char* buffer, char split, char* p1, char* p2)} Split
the string in \texttt{buffer} by the first occurennce of \texttt{split}. Store
the first part in \texttt{p1} and rest in \texttt{p2}. Returns 1 in the case of
failure, 0 otherwise.

\subsubsection{Util\_max} \label{apiutilmax}
\cdoc{#define Util_max(a, b)} This macro returns the greater of a and b. Because
this is a macro, it is up to the user to ensure only valid types are passed to
it.

\subsubsection{Util\_min} \label{apiutilmin}
\cdoc{#define Util_min(a, b)} This macro returns the lesser of a and b. Because
this is a macro, it is up to the user to ensure only valid types are passed to
it.

\subsubsection{Util\_inRange} \label{apiutilrange}
\cdoc{#define Util_inRange(a, x, b)} This macro returns the value \texttt{x}
bounded by \texttt{a} and \texttt{b} such that a $a \leq x \leq b$. If $x < a$
then \texttt{a} is returned. Likewise, if $x > b$ then \texttt{b} is
returned. Otherwise, \texttt{x} is returned unmodified. Because this is a macro,
it is up to the user to ensure only valid types are passed to it.

\subsection{Timer} \label{apitimer}
\subsubsection{Timer\_new} \label{apitimernew}
\cdoc{Timer* Timer_new(void)} Return a new timer object

\subsubsection{Timer\_getDelta} \label{apitimerdelta}
\cdoc{double Timer_getDelta(Timer* tm)} Return the time in seconds since the 
last call to \func{Timer\_getDelta}, or since the time the timer object was
created or reset

\subsubsection{Timer\_getTotal} \label{apitimertotal}
\cdoc{double Timer_getTotal(Timer* tm)} Return the time since the timer was 
created or last reset.

\subsubsection{Timer\_reset} \label{apitimerreset}
\cdoc{void Timer_reset(Timer* tm)} Resets the timer such that future 
\func{Timer\_*} calls are with respect to the current time and no longer the 
time of the timer's creation

\subsubsection{Timer\_destroy} \label{apitimerdestroy}
\cdoc{void Timer_destroy(Timer* tm)} Destroy a timer object and free the memory
associated with it.

\subsection{PID} \label{apipid}
\subsubsection{PID\_new} \label{apipidnew}
\cdoc{PID* PID_new(double sp, double p, double i, double d)} Return a new PID
associated with the given p, i, and d values given as well as the initial set
point `sp'

\subsubsection{PID\_start} \label{apipidstart}
\cdoc{double PID_start(PID* pid, double pv)} Start/restart the controll with
the initial process value bbpv. Returns only the proportional part of the error.

\subsubsection{PID\_update} \label{apipidupdate}
\cdoc{double PID_update(PID* pid, double pv)} Return a new output value for the
given process value pv.

\subsubsection{PID\_resetIntegral} \label{apipidresetint}
\cdoc{void PID_resetIntegral(PID* pid)} Reset the integral component of the PID.

\subsubsection{PID\_setSetPoint} \label{apipidsetsp}
\cdoc{void PID_setSetPoint(PID* pid, double sp)} Give a new set point for the
PID. You should also restart the PID at this point unless you know exactly what
you are doing.

\subsubsection{PID\_setCoefficients} \label{apipidsetcoff}
\cdoc{void PID_setCoefficients(PID* pid, double p, double i, double d)} Set new
coefficients for the controller

\subsubsection{PID\_destroy} \label{apipiddestroy}
\cdoc{void PID_destroy(PID* pid)} Free the memory associated with the
controller.

\subsection{Stack} \label{apistack}
\subsubsection{Stack\_new} \label{apistacknew}
\cdoc{Stack* Stack_new(void)} Return a new stack object

\subsubsection{Stack\_push} \label{apistackpush}
\cdoc{void Stack_push(Stack* stack, void* v)} Push the value v onto the given
stack

\subsubsection{Stack\_pop} \label{apistackpop}
\cdoc{void* Stack_pop(Stack* stack)} Pop the first value off of the top of the
stack and return it

\subsubsection{Stack\_top} \label{apistacktop}
\cdoc{void* Stack_top(Stack* stack)} Return the first value off the top of the
stack without removing it.

\subsubsection{Stack\_getSize} \label{apistacksize}
\cdoc{int Stack_getSize(Stack* stack)} Returns the number of elements in the
stack

\subsubsection{Stack\_destroy} \label{apistackdestroy}
\cdoc{void Stack_destroy(Stack* stack)} Destroy the stack and free the memory
associated with it

\subsection{Task} \label{apitask}
\textit{This is a rather specialized and technical component. If someone would like it documented let me know}
