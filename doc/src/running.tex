
\newpage
\section{Running Seawolf Software} \label{running}
This section will address running Seawolf software in general and then running
it for Seawolf III in particular.

\subsection{Database Configuration} \label{runningmysql}
The SeaSQL component of \libseawolf{} requires a MySQL database to store data so
one must be running for any Seawolf applications to start. If you intend to run
MySQL on your personal computer then you will need to install and run it. For
Debian GNU/Linux derived systems (including Ubuntu, Kubuntu, Knoppix, etc.) this
can be done by running the command \texttt{apt-get install mysql-server} as the
root user. If you are not running one of these system, consult the documentation
for your system for how to install and run a MySQL server. A minimal
configuration will require that at least a database be created for Seawolf to
use. This can be done with the command \texttt{mysql -u root -e `CREATE
  DATABASE seawolf;'}. 

\subsection{Seawolf Configuration} \label{runningconfig}
You will now need to ensure that the Seawolf configuration file is properly set
up. This file is located at \file{CODE2009/conf/seawolf.conf} with respect to
the root directory of the SVN. Check that the database specified in this file
matches the name of the database you have created (seawolf). Most defaults
should be fine but you can adjust as necessary. See
\interlink{formatsseawolfconf}{Seawolf Configuration File}.

\subsection{Updating and Building} \label{runningupate}
The Seawolf software evolves quickly so it is important to insure that your code
and binaries are always up-to-date. To update Seawolf from a UNIX-like system
begin by opening a terminal and change your working directory to the one
corresponding to \folder{CODE2009} in the SVN. Next, run the command \texttt{svn
  update} to update your copy of the source code to the most recent version
available in version control. 

After an update from SVN, your binaries may be out of date. To update your
binaries start by moving into the \folder{libseawolf} folder and issuing the
command \texttt{./make}. This will update \libseawolf{} and it is important that
it be rebuilt before anything else, because the other components depend on
it. Next move to the \folder{serial} folder and execute \texttt{./make} and do
the same for the \folder{applications} directory. Once these three components
are up-to-date you are ready to run the Seawolf code.

\subsection{Execution Environment} \label{runningenviron}
Before any Seawolf applications can be run a proper execution environment must
be set up. To do this the hub server (\file{hub}) must be run along with the
serial manager (\file{serialapp}). The hub server must be run first because all
Seawolf applications depend on it, \file{serialapp} included. To launch the hub
server change to the \folder{libseawolf} folder and execute \texttt{./hub}. If
the hub server is running corretly then no output will be generated and you will
not be returned to your terminal prompt. 

Now open a new terminal window and navigate to the \folder{serial} directory and
execute \texttt{./serialapp}. This should start generating output and you should
see it attempting to identify serial devices. Once all connected devices are
identified the execution environment is ready.

\subsection{Running an Application} \label{runningapp}
Begin by opening a new terminal window and navigating to the
\folder{applications} directory. From here you can can execute
\texttt{./bin/\bracket{appliction}} where \texttt{application} is the name of
the application you wish to run. This is all there is too it. If you need to run
another application at the same time just repeat the process with a new terminal
window.

\subsection{Possible Errors} \label{runningerror}
\subsubsection{./someapp: error while loading\ldots{}}
\begin{lstlisting}
% ./someapp
./someapp: error while loading shared libraries:   \ 
    libseawolf.so: cannot open shared object file: \
    No such file or directory
\end{lstlisting}

This indicates that the \libseawolf{} library file could not be located. To let
the system know where it is you can set the \texttt{LD\_LIBRARY\_PATH}
environment variable. Example, \texttt{LD\_LIBRARY\_PATH=../libseawolf/ ./someapp}

\subsubsection{Could not connect to hub/repeater\ldots{}}
\begin{lstlisting}
% ./someapp
Could not connect to hub/repeater server: Connection refused
Exiting
\end{lstlisting}

This indicates that the hub server could not be contacted. To start the hub
server see \interlink{runningenviron}{Execution Environment}. If you are sure
you have started the hub server correctly then see
\interlink{runningconfig}{Seawolf Configuration} to ensure you have configured
the correct host name for the server.

\subsubsection{Error connecting to MySQL database: Can't connect\ldots{}}
\begin{lstlisting}
% ./someapp
Error connecting to MySQL database: Can't connect to \
    local MySQL server through socket                \
    '/var/run/mysqld/mysqld.sock' (2)
Exiting
\end{lstlisting}
This indicates that Seawolf couldn't connect to the configured MySQL server. See
\interlink{runningmysql}{Database Configuration} to ensure you have MySQL
properly installed and see \interlink{runningconfig}{Seawolf Configuration} to
check that your Seawolf configuration matches your MySQL configuration.

\subsection{Running on Seawolf III} \label{runningsw3}
The previous material has covered running Seawolf software in general and now we
will considered running this software on Seawolf III in particular.

\subsubsection{Network Connection} \label{runningsw3net}
Make sure that the netbook in Seawolf III is running and has the Ethernet cable
and USB hub connected properly connected. Next, connect the system that will be
used to remotely control Seawolf III by plugging in the other end of the
Ethernet cable to this system. You'll now need to configure the Ethernet port on
your computer. The remote control system should have IP address 10.17.0.1 and
netmask 255.255.255.0. How this is configured will dependin on your operating
system. For Linux systems, this can be done by running the command
\texttt{ifconfig eth0 inet 10.17.0.1 netmask 255.255.255.0 up} as the root user,
where eth0 is the particular Ethernet device.

\subsubsection{Connecting to Seawolf III} \label{runningsw3ssh}
You'll now need to SSH to the netbook in Seawolf III. Microsoft Windows users
can use PuTTY (\url{http://www.chiark.greenend.org.uk/~sgtatham/putty/} to do
this, while users of UNIX-like systems can simply use the command line SSH
client. Using either method, attempt to SSH to system 10.17.0.2 with username
seawolf. For users of UNIX-like systems this would like look this, \texttt{ssh
  seawolf@10.17.0.2}. Enter the password and you should find yourself logged
into the netbook.

\subsubsection{Starting Screen}
Screen is a command for UNIX-like system that allows a user to run multiple
command line interfaces from one actual command line session. Screen allows you
to start multiple shell instances, switch between them and detach from an active
session to resume it later. Once you log into the netbook begin by typing
\texttt{screen} at the prompt and hitting return. This should start screen and
you will be presented with a command line interface just as before. Start off by
create a few new command line interfaces. You can do this by hitting Ctrl-h and
then hitting c (short for create). You'll find that all commands in screen start
with Ctrl-h which acts as a command prefix. You'll want three or four command
line interfaces so repeat the Ctrl-h, c command as many times as necessary. We
will use these numerous command line interfaces to run the applications outlined
in the previous sections. You can switch between your first 10 screen command
line interfaces by hitting Ctrl-h followed by a number 0-9 on your keyboard.

\subsubsection{Running the Programs}
Switch to the first interface (Ctrl-h, 0) and then execute \texttt{cd
  seawolf3/\newline{}libseawolf}. This has changed your current directory to
\folder{seawolf3/libseawolf} which is where the hub server is located. To run
it, execute \texttt{./hub}. Now switch to the second screen interface (Ctrl-h,
1) and change your working directory to \folder{seawolf3/serial} (\texttt{cd
  seawolf3/serial}). Then execute \texttt{./serialapp} and wait for all serial
devices to be identified. After this you can start running applications. Switch
to the third screen interface and move to the \folder{seawolf3/applications}
directory. You can then execute any program in \folder{bin} by executing
\texttt{./bin/someprogram}.

\subsubsection{If you get disconnected}
If your SSH connection ever drops while you're connected you can easily
reconnect and resume your previous session. Begin by SSHing back into the
netbook and once you are at the command line you should be able to run
\texttt{screen -dr} and reconnect to your last screen session.
